% !TeX spellcheck = en_GB

	As explained in the section \ref{section:objectives}, this work can be defined as the very first approach of one of the parts of a much larger and ambitious project. So, it is easy to understand the future work is very extensive in order to fulfill the final requirements. However, we propose a couple of details and considerations that we thought a good idea when trying to implement our approach.
	
	One of the most important aspects when working on machine and deep learning problems is the quality and amount of data. Finding these resources is actually a really tough and long task. This is one of the first points that should be taken into account in order to go further in this investigation line. Researchers from the EMPATIA-TC project are currently working to solve this problem by trying to develop a database that collects data related to stimuli, labels for emotions and reactions of victims \cite{UC3M4SafetyTeam2018} in situations of violence against women. However, within the scope of this work, since the violence databases that are available do not focus on this task or are little populated, it would be interested to make a public dataset that allow to fully adapt the problem to the gender-based violence, including a wide range of sounds that may really happen in this kind of situations. This could be done in a much less complicated way without recording any victims, for example, by collecting audio data that strictly belong to this field from public websites and subject them to a proper labeling process.
	
	More work on trying to establish a subjective definition of violence that could be implemented is also possible. For our case, considering the way we chose, explained in subsection \ref{subsection:our-point-of-view}, it could be designed a better implementation of a system that eases the contribution of the victim to the whole learning process so it is not so annoying to be selecting all the categories, or, maybe, an automatic system in which the victim does not need to directly participate.
	
	Related to the feature extraction process, more studies could be done in order to finally select a feature space that totally adapts the problem and makes available the separability between violent and non-violent events. If we consider the embeddings extracted with \acrshort{vgg}ish, a more immediate research line could be to try different values of the parameters related to the final shape of the embedding and overlapping between rows, as mentioned in subsection \ref{subsection:data-access}, so as to see the different results when extracting embeddings from audio data.
	
	 A good study to dive into is to improve the system used in the learning process by trying to apply new methods and algorithms or just taking more advantage of the ones already used, explained in subsection \ref{section:methodology} and chapter \ref{chapter:experiments}. For example, adapting the \acrshort{cnn} architecture model to the available data or improving the performance of the \acrshort{lstm}. In this last aspect, we did try to study a little the possibility of implementing an attention-based approach mixed with \acrshort{lstm}, following the idea proposed in \cite{Wang2016a} for sentiment analysis, but it was going to become very complex and out of our initial scope.
	 
	 Finally, the implementation of a model for multilabel\footnote{Multilabel classification differs from multiclass classification in the way the samples are labeled. The former treats samples that belong to more than one class at the same time, and the latter, the one implemented in this project, works with samples that belong to just one class.} classification. This is actually a very important topic that could increase the scope of the problem, also paying attention to its real world application. We just worked with monolabel samples from the dataset, which supposed a big limitation on the process of class selection. A bit of research was also done in order to make an attempt on this task by trying to use \acrfull{cc}, which treats the multilabel problem as a combination of several monolabel classifications \cite{Read2011}, but we did not get any further.
	 