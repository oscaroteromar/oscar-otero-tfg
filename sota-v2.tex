% !TeX spellcheck = en_GB

\subsection{Features and methods}
	
	As in many fields, the frontier between features and methods used for audio tasks have been blurred during the last decades. Usually, a problem is addressed with a pre processing period of the data and then the model or method is implemented. Right now, these two stages are sometimes maintained but also have been mixed or changed depending on how the algorithm used works. In this chapter, we are going to try to explain these difference between features and methods and try to separate them in order to ease its understanding.

\subsubsection{Features}

	In every machine learning or pattern recognition task, for the system to be able to infer and extract conclusions from the input given, it is necessary a pre processing period in order to make some transformations to the data so this can be readable by the model. This stage is known as feature extraction and the goal is to convert the original information into a set of of values or vectors that characterize the data regarding some desired properties \cite{Giannakopoulos2014}.
	
	There are several ways that allow us to perform this processing stage. The most common and basic one consists on extracting features that are really closely related to the original signal which are called \acrfull{lld}  \cite{Amatriain2004}. These are computed by performing some mathematical operations or formulas to the original data that can be considered rudimentary when comparing with other techniques. However, they are really extended and still in use nowadays \cite{Marr1982}. 
	
	\todo{Include small definition short-term features and long-term features?}
	In the audio field, there are two types in which all \acrshort{lld} can be grouped into. One of them is for the features that have been computed by considering the audio signal in its original form in the recording, i.e., in time-domain, that is the reason why they are known as time-domain audio features. The other case refers to those characteristics that are obtained from the signal after been transformed to frequency domain. These are commonly known as frequency-domain or spectral audio features. For the procedure of feature extraction, the signal is usually divided into frames that can be overlapped by using a sliding window, so the calculations are done per frame, obtaining a final matrix with size $number\ of\ frames \times number\ of\ features$ \cite{Giannakopoulos2014}. It must be taken into account that the final application of the whole system is going to be fundamental at the time of deciding which features must be computed. For example, not the same features are used for speech recognition than for musical information retrieval. In table \ref{table:6}, a summary of most used time and spectral features is included.
	
	% Table of time domain features
	\begin{table}[h!]
		\begin{center}
			\centering
			\begin{tabular}{|| m{9em} | m{24em} ||}
				\hline
				\begin{center}\textbf{Feature}\end{center}& \begin{center}\textbf{Description}\end{center} \\
				\hline\hline
				\multicolumn{2}{||c||}{\textbf{Time}} \\
				\hline
				Energy Entropy & It is a useful to detect sudden changes from the energy of a signal. To calculate this value for a certain subframe, it is necessary to first compute the normalized energy of the subframe with respect to all the frames energy \cite{Giannakopoulos2006}. \\
				\hline
				Short time energy & It is the energy for a short segment of signal. It is normally used in speech tasks in order to identify voiced form non-voiced fragments \cite{Garcia-Gomez2016}. \\
				\hline
				\acrfull{zcr} & This can be defined as the number of times the amplitude of the signal crosses the zero line, i.e., changes from negative to positive. It is computed by the number of zero-crossings by the amount of samples in the frame \cite{Giannakopoulos2006}. \\
				\hline
				\multicolumn{2}{||c||}{\textbf{Frequency}} \\
				\hline
				\acrfull{sf} & It is computed to measure the spectral changes between two successive frames. To do so, the difference is calculated through their squared spectra coefficients normalized \cite{Giannakopoulos2006}. \\
				\hline
				Spectral Rolloff & This represents the skewness of the shape of the spectrum by given the frequency below which a concrete percentage of the magnitude distribution of the frequency transform is concentrated \cite{Garcia-Gomez2016}. \\
				\hline
				\acrfull{sc} & This is a measure related to the spectral position. It is defined as the center of gravity of the spectrum, i.e., \doubt{it indicates how high the spectrum values are on average} \cite{Giannakopoulos2006}. \\
				\hline
				\acrshort{mfcc} & It is a feature that it is widely used and it has given such good results, mainly within the speech recognition field because it interprets the frequency bands in a very similar way to human perception. It is computed from the \acrshort{stft}.\cite{Garcia-Gomez2016}. \doubt{A detailed explanation can be found in appendix \ref{}}. \\
				\hline
			\end{tabular}
		\end{center}
		\caption{Time-domain and spectral audio features}
		\label{table:6}
	\end{table}

	Then, a typical procedure would first divide the signal into frames and performed the desired mathematical and statistical transformations to obtain a \acrshort{lld} vector per sample in order to pass these input data to a statistical model that will be trained with the objective of summarizing the properties of the signals and develop a classification rule so as to be able to assign a faithful category to a new unlabelled observation \cite{Stowell2015}.
	
	In some cases, the mentioned descriptors are not enough to model data, just because they cannot achieve an enough meaningful representation of the original information for the system to learn from it. Also, as mentioned above, the selection process of this type of features plays a significant role in the final output so that the criteria of the designer could be considered a limitation for the classification task \cite{Grill2012}. For this reason, a new strategy of addressing feature extraction process has appeared based on the idea of finding more specialized properties by using specific engineering algorithms that explores the given data in order to find non-human recognizable patterns. The techniques that allow to perform this task are usually called automated feature generation algorithms \cite{Pachet2009}.
	
	In order to compute these high-level features, there is not just one standardized machine model that allow to compute them all in a particular form. The calculation method, nevertheless, differs from one approach to another. One example could be the combination of low-level data with high-level semantic descriptors that consists on the inference of diverse musical dimensions by using support vector machines so as to find similarities among music genres. Particularly, they consider the output probabilities of the \acrshort{svm} classifier as a high-level feature space in which the distance between samples from different classes can be measured \cite{Bogdanov2011}. Other application example could be building high-level feature detectors for image recognition by using unlabelled data to feed a \acrshort{nn} model \cite{Le2013}. 
	
\subsubsection{Classification models}
	
	Several algorithms have been designed in order to perform the classification task by finding data patterns from input data features. \doubt{We are going to include a small explanation of some of the most common used in the literature and go deeper in those we have used}. Some of the most common and base methods to tackle the classification task are \acrshort{knn}, \acrshort{svm} and also \acrshort{gmm} \cite{Fu2011}.
	
	In the case of \acrfull{knn}, if there is a set of samples ${x_1, ..., x_n}$ that belong to a metric space $X$ whose labels are ${\theta_1, ..., \theta_n}$ respectively, if a new sample $x$ comes in, it will be categorized by following a majority vote process of the nearest neighbours considering the distance in the space $X$. A \doubt{engagement relationship} must exist in the number of nearest neighbours since it is desirable to be large for the voting process to have more participants and minimize the probability of misclassification and, also, a small value with respect to the total number of samples so the nearest points are close enough to the new observation \cite{Cover1967}. \todo{Complete with example in the ASC or AEDC approaches}
	
	In the case of \acrfull{gmm}, it is a soft clustering method. It is commonly used to derive global statistical properties from the feature vectors of the sample data. Specifically, considering $K$ clusters, it defines a group of feature vectors from training data and belonging to a certain category $q$ modelled as a multivariate normal distribution $N(\mu_k, \Sigma_k)$ that is weighted by the probability $w_k$ of a particular of a certain observation to belong to cluster $k$. After learning a global model for the training data $M_q={w_k, \mu_k, \Sigma_k}$, a new observation will be categorized by applying the maximum likelihood criterion \cite{Stowell2015}.
	
	However, the one that have been generally more successful is the \acrfull{svm} classifier. There have been published plenty of works that use this technique for audio classification tasks \cite{Jiang2005} \cite{Geiger2013} \cite{Barchiesi2015}. It is commonly known to be a binary classifier but, nowadays, there have been some implementations that allow to use it in a multiclass problem. This algorithm makes the classification by taking into account the two observations that are more likely to be misclassified and draw a hyperplane as the optimal frontier between the two classes \cite{Fu2011}. Since this technique has been used for the experiments in this work, a more detailed explanation is included in section \ref{section:models}.
	
	As a final kind of method, it should be mentioned all the techniques or algorithms that have been developed based on the idea of \acrfull{ann}. This concept was animated by human biology and how the neurons communicate among them inside a brain in order to interpret the input data the human senses collect. 
	
	The \acrshort{ann} intention is exactly to transform input data into a desired output space so it can understandable by a machine. A basic architecture is usually composed by an \textit{input layer} of neurons, a certain number of \textit{hidden layers} and an \textit{output layer}. The model receives some input, typically a feature vector, and pass it trough the hidden layers in order to perform some transformations. The neurons of a layer are completely connected to the neurons of the next layer but neurons in the same layer are totally independent. For this case, we can say that they are \acrfull{fc} layers. Finally, the output layer gives the final outcome of the network \cite{Kwon2011}. This is just the basic and initial nature of the idea but it was implemented in many different ways depending on the input data and the desired task.
	
	\todo{Include image of FCC?}
	


	
	
	
	

	
	
	
	
	
	
	
	