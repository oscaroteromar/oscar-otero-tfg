% !TeX spellcheck = en_GB
% Methods text

\todo{CPM: te recomiendo partir este capítulo en dos porque tienes que traerte lo de las CNN desde el capítulo 2 y además te falta completar las CNN. Quedará demasiado largo. El nuevo capítulo incluiría 3.2.3 y 3.3 al completo. }

\todo{CPM: el primer capítulo lo puedes llamar Resources and Methods for Acoustic Violent Detection y el segundo, Our approach for Acoustic Violent Detection. Sería bueno que este segundo acabase con un diagrama de bloques de todo el proceso.}

\todo{CPM: cuando tengas los dos capítulos añade un párrafo al comenzar cada uno de ellos que diga cómo está organizado.}

\section{Database: AudioSet}
\label{section:audioset}

\subsection{What is AudioSet} \todo{How cite? CPM: como misc, por ejemplo en la página web y con el artículo que la describe}

	Audio Set can be described as a large-scale sound dataset with the goal of putting the availability of audio and image data on the same level. It is composed by a huge variety of 2,084,320 manually-annotated audio events in video format and it is organized by following an ontology formed by 632 different audio classes. The data has been extracted from YouTube videos and the labelling process has been based on diverse factors such as metadata, context and content analysis. It has been developed by Google with the purpose of producing an audio event recognizer that can be applied to plenty of acoustic situations coming from the real world \cite{Gemmeke2017}.
	
	Due to some limitations found during the building process that will be explained along the section, the current final release of the dataset is composed by 1,789,621 segments, where each video usually last 10 seconds which results in a total duration of 4,971 hours of video and audio. After executing the selection process in which the different labels were populated with the final corresponding segments, a total of 527 classes were gathered, out of which 485 counted with at least 100 samples.
	
\subsection{Ontology}
\label{subsecition:ontology}

	In order to put this dataset together the events have been organized in an abstract hierarchy. This is composed by higher-level classes which describe a certain type of sound event and also act as parents of other labels that refer to more specific events. With this purpose, the relationship among different classes needs to be non-exclusive, so labelling similar audio events may result into a more general class, the parent, if there was ambiguity. This is also helpful for labellers due to group the clips in an easier and faster way.
	
	The Audio Set Ontology was collected considering some fundamental guidelines explained below:
	% List
	% !TeX spellcheck = en_GB
\begin{itemize}
	\item A complete collection of all desirable labels was prepared so that it can be used to define sound events from real-world aural data.
	\item When labelling an audio event the result must match the criteria of a common listener.
	\item Different categories should be easy to distinguish by an ordinary listener. In the case that two different labels do not satisfy this requirement, these should be merged. With this condition, the plethora of possible labels remains limited.
	\item The distinction of two different classes must be done by relying just on the audio, it cannot be accompanied by image or visual information.
	\item The hierarchy should not be very deep by keeping the number of children per parent class to no more than 10. This also eases the annotation labour. 
\end{itemize}

	
	It is easy that an ontology of this volume gets leaned or biased towards a particular direction due to several factors, such as the subjectiveness of its creators or the selection of the initial set of classes used when starting to label. With the intention of generating a primary list that covers a wide range of audio events in an objective way, the researchers decided to apply an impartial, web-scale text analysis from the very beginning. They agreed on detecting hyponyms of the word "sound" by utilizing a modified version of the famous technique called \textit{Hearst patterns} \cite{Hearst1992}, a method proposed to automatically acquire hyponymy lexical relations from unrestricted text. As a result, an enormous collection of terms came up. This was filtered by considering how well these terms represented audio, i.e. by combining together the global frequency of occurrence and how exclusively these are recognized as hyponyms of "sound" instead of other terms. As an output of the final process, a list of 3000 terms was obtained.
	
	With respect to the hierarchical relation among categories, it was constructed by the authors with the main intention that this satisfied their human comprehension of the sounds. Event though this is a subjective manner, it also makes sense since this is how the hierarchy best performs its labour on helping human labelling.
	After all the organization process, the model is not based on a strict organization as a singular node can appear in many different locations, i.e., a single node can be child of different parent nodes. As we said before, the final result is composed by 632 audio event labels and, in the hierarchy, the deepest class is six level away from the top parent. Figure \ref{fig:mesh1} shows the nodes that belong to the two top layers of the ontology.
	
	% Figure of ontology
	\begin{figure}[h]
		\centering
		\captionsetup{justification=centering}
		\includegraphics[scale=0.21]{audioset-ontology}
		\caption{First two layers of Audio Set Ontology \cite{Gemmeke2017}}
		\label{fig:mesh1}
	\end{figure}
	
	This whole structure is offered to any user as a file in JSON format. A couple of fields have been included for each label in order to describe its meaning and make clear its position within the hierarchy. A description for all of these can be found in table \ref{table:2}. 
	
	% !TeX spellcheck = en_GB
% Ontology-fields table

\begin{table}[h!]
\begin{center}
	\begin{tabular}{|| m{7em} | m{22em} ||}
		\hline
		ID & This field includes the Knowledge Graph \acrfull{mid} that best describes the sound or its source. It is used as a primary identifier for the class. \\
		\hline
		Display name & Short name formed by one or two words that identifies the audio class. It sometimes includes a small alternative separated by a comma so it does not feel ambiguous. \\
		\hline
		Description & One or two explaining sentences so the meaning of the category is more defined. These can be extracted from Wikipedia or WordNet. \\
		\hline
		Examples & At least one example of the label is provided as a URL of a YouTube video. \\
		\hline
		Children & An array filled by the \acrfull{mid}s from all immediate children of the class. \\
		\hline
		Restrictions & It specifies if the category in question either has been discarded or there are no audio clips under it. \\
		\hline
	\end{tabular}
\end{center}
\caption{Fields per category in the ontology.}
\label{table:2}
\end{table}
	
	For the field \textit{ID}, the identifiers are known as \acrfull{mid} and belong to the \textit{Knowledge Graph} designed by Google \cite{Singhal2012}. This is a knowledge base that Google services use to improve the quality of its search results and it is composed with information extracted from a wide variety of sources. The \acrshort{mid}s are the identifiers of the different elements that belong to this huge dictionary. For instance, the \acrshort{mid} of the word "Speech" is "/m/09x0r". Another field that deserves a special explanation is the one corresponding to \textit{Restrictions}. Within all the categories of sound events, there are two flags that indicate an exclusive behaviour that differ from a typical label: "blacklisted" and "abstract". The former refers to a class that has been hidden from labellers due to its confusing meaning. The latter has been used for those classes that are just employed as intermediate nodes in order to provide a better grouping inside the organization, and are not expected to be used in the implementation tasks. In total, out of the 632 categories, 56 have been categorized as "blacklisted" and 22, as "abstract".
	
	
%\subsection{Data}
%\label{subsection:data}
%\todo{CPM: esta subsección no sé si tiene mucho sentido. Por uan parte el primer párrafo parece hablar de Data formats (que luego está repetido en la subsección siguiente) y el segundo de la cantidad de datos que quizás es algo que debieras haber dicho ya al principio de la sección anterior.}
%
%	The different YouTube videos that constitute the dataset are included in a \acrshort{csv} file in which each row comprises the video identifier, the start and end time of the audio event within the video and the ontology labels that the certain clip belongs to. All the video segments have a length of 10 seconds maximum, except from those that the original video is shorter, then the whole thing will be considered as the an audio event. \todo{Include any explanation about ratings? CPM: no sé a qué te refieres...}
%	
%	
	
\subsection{Data access}
\label{subsection:data-access}

	The data can be obtained through the website \cite{SoundUnderstandinggroup2017} in two different formats:
	
	\begin{itemize}
		\item Files in \acrshort{csv} format that include for each video segment its YouTube video ID, start time, end time and the one or more labels it belongs to. 
		\item Instead of the audio files themselves, they provide already extracted audio features for each segment in compressed files that are available in \acrfull{tf} \cite{GoogleResearch2015} record files that can be easily downloaded.
	\end{itemize}

	Together with these two options of obtaining the data, a neural network model is also provided in the database website which can be used to extract embeddings as high-level features. It is called \textit{\acrshort{vgg}ish} \cite{Hershey2017} and it has been pre trained on a preliminary version of the database YouTube-8M \cite{Abu-El-Haija2016}. A more detailed explanation of this system is later included in \ref{subsection:vggish}.
	
	Based on the first choice, we did some attempts of downloading a couple of videos from YouTube using the information given in the \acrshort{csv} files. Then, we decided to try to extract our own embeddings features by running the \acrshort{vgg}ish network with the downloaded videos in order to see how they look and check their behaviour on a classification task. Also, for the second option, we downloaded the provided embeddings already extracted by the Google developers and see if they were also useful for our problem. A little experiment explained in \ref{subsection:exploring-differences-between-two-types-of-data-access} were implemented so as to check if the two manners of obtaining the embeddings were equivalent.
		
\section{Feature extractor}
\label{section:feature-extractor}

	\acrshort{vgg}ish model design was based in an already existed network called \acrshort{vgg}. In order to describe the embedding features extractor we used, it is first needed to make a clear explanation of what a \acrlong{cnn} is and how the \acrshort{vgg} model is implemented. Then, we can move to analyze the differences and changes included in \acrshort{vgg}ish. 

\subsection{\acrlong{ann}s (\acrshort{ann}) and \acrlong{cnn}s (\acrshort{cnn})}
\label{subsection:ann-cnn}
	
	Before going deep in the description of \acrshort{cnn}, the core idea of \acrshort{ann} must be explained. As it was already mentioned at the end of subsection \ref{subsubsection:classification-models}, these are based on the communication way that the neurons follow inside a brain to interpret the stimuli they receive. This is actually the hidden concept behind the \acrshort{ann} intention, which is transforming input data into a desired output so that it can be used to perform a given task. %it can understandable by a machine. 
	
	A basic architecture is usually composed by an \textit{input layer} of neurons, a certain number of \textit{hidden layers} and an \textit{output layer}. The model receives some input, typically a feature vector, and pass it trough the hidden layers in order to perform some transformations. The neurons of a layer are completely connected to the neurons of the next layer but neurons in the same layer are not usually connected. For this case, we can say that they are \acrfull{fc} layers. Finally, the output layer gives the resulting outcome of the network \cite{Kwon2011}. This is just the basic and initial nature of the idea but it was implemented in many different ways depending on the input data and the desired task.
	
	Within the different possibilities of designing an artificial neural network, there are two main options. The selection of one of them depends on the complexity of the problem, the format of the input data and how it is combined with other techniques. One of them covers all the models that are structured with a shallow architecture. An example of this would be a basic type of network known as \acrfull{mlp} that follows the structure explained for \acrshort{ann} with \acrlong{fc} layers but composed just by one hidden layer. This type of systems have performed well in many tasks but they have shown some limitations in more complicated applications \cite{Deng2014}.
	
	When working in real-world cases such those that involve natural signals, such as human voice, natural sounds and visual scenes, for example, it is needed the development of a deeper architecture in order to extract and understand the complexity inside this type of data so as to build. For example, visual and speech recognition tasks are exploited with system that follows the hierarchical transformations that the humans perform when understanding this type of data. reliable and useful internal representations. This type of models are collected inside the \acrfull{dnn} field. \acrshort{dnn} has been one of the most studied fields since it was proposed in 2006 \cite{Deng2014}. In figure \ref{fig:mesh48}, it can be found a comparison between shallow and deep models.
	
	\begin{figure}[ht]
		% Whole figure
		\captionsetup{justification=centering}
		\begin{subfigure}[b]{\textwidth}
			% Start with figure wav
			\centering
			\captionsetup{justification=centering}
			\includegraphics[width=0.45\linewidth]{shallow}
			\subcaption{Shallow architecture}
		\end{subfigure}
		\vskip\baselineskip
		% Start with figure tfrecord
		\begin{subfigure}[b]{\textwidth}
			\centering
			\captionsetup{justification=centering}
			\includegraphics[width=0.6\linewidth]{dnn}
			\subcaption{Deep architecture}
		\end{subfigure}
		
		\caption{Comparison between shallow networks and deep neural networks \cite{Di2018}}
		\label{fig:mesh48}
	\end{figure} 
	
	% !TeX spellcheck = en_GB
% CNN explanation

\label{section:cnn}

	One of the most explored implementations is the one related to \acrfull{cnn}. This maintain the basic idea of receiving some inputs and apply a dot product operation to them before subject the data to a non-linear function but with the difference that input data are expected to be images \cite{Karpathy2016}. The architecture of a \acrlong{convnet} is based on the organization of the Visual Cortex in the human brain. This means that single neurons responds to some stimuli only in a certain region of the visual field, the Receptive Field. Then, a collection of these fields overlap with each other so as to cover the full visual space \cite{Saha2018}.
	
	About the input data, it is introduced as a matrix instead of as a row vector. Then, the architecture of the layers adapt to the fact that input data are images. They do not consist of a single row of neurons, but they have a shape constituted by three dimensions: heigh, width and depth \cite{Karpathy2016} \todo{include image?}. In this way, the net can capture the spatial and temporal relations present in an image by applying some filter functions. With this construction, the model can be trained with a smaller number of parameters and also reuse the weights.
	
	A \acrlong{convnet} can be described as a sequence of layers in which every layer converts a volume of activation results from the previous layer to another by computing a differentiable function. The architecture of a \acrshort{cnn} is always composed by three type of distinguishable layers which are \textit{Convolutional}, \textit{Pooling} and \textit{\acrlong{fc}}. A normal architecture will follow the next succession of layers \cite{Karpathy2016}:
	
	\begin{itemize}
		\item INPUT: this represents the image at the entrance of the whole model. It commonly is a 3\acrshort{d} matrix , one dimension per channel of colour, and contains the values of the original pixels.
		\item CONVOLUTIONAL: considering the neurons that are connected to a region of the input image, this layer computes their output by performing a dot product between the weights and the value of the pixels. The depth of the resulting data will coincide with the number of filters used.
		\item POOL: this performs a fixed operation that consists of a downsampling of the spatial dimension. It typically reduces by half the height and the width of the input image.
		\item \acrshort{fc}: this is the final layer which gives the output of the whole net. It is a vector-shape layer and is completely connected to the previous layer. In a classification task the number of neurons matches the amount of classes of the problem.
	\end{itemize}
	
	Note that this list represents the order of the layers in the architecture, however, the number of layers of each type vary depending on how the model is designed and how deep it is desired to be. For example, some examples of deep architectures can be found later in section \ref{subsection:vgg}, in table \ref{table:3}.
	
	With this building, a \acrlong{convnet} passes the information of the pixels layer by layer while reducing the input images in a way that is much easier to process. This is performed without missing any feature since all of them are essential to compute a good prediction outcome.
	
	In the convolutional layer, a convolution operation is performed for each small region of the input image. The element that carries out this operation is called \textit{kernel} or \textit{filter}. This partial region that the kernel covers in the input data is set by deciding the height and the width as a designing parameter. About the depth, it must match the depth of the input data so the filter can slide trough the image along its 2\acrshort{d} and obtaining the results of the convolution operation by computing between the pixels inside the region and the values of the filter. An example can be observed in figure \ref{fig:mesh11}. For each filter, a 2\acrshort{d} activation map is obtained after subjected the result of the convolution to a non-linear operation as the \acrfull{relu} function. Finally, we will pack all these in a unique volume that will consists in the output of the convolutional layer \cite{Karpathy2016}. For example, if we have a number of 64 filters, we will get then a group of 64 2\acrshort{d} activation maps that will be the input of the next layer. Traditionally, it was the first convolutional layer the one that was responsible of capturing low-level features related to edges or colours in the image. The upper layers were the ones in charge of extracting high-level features so the network was able to understand the image similar to how the humans do \cite{Saha2018}.
	
	\begin{figure}[ht]
		\centering
		\captionsetup{justification=centering}
		\includegraphics[scale=0.6]{movement-of-kernel}
		\caption{Movement of the kernel represented as a red block along the width and the height of the original input image. It is clear that it occupies the whole depth of the input image. The arrow indicates an approximation of the movement that the filter follows \cite{Saha2018}.}
		\label{fig:mesh11}
	\end{figure}

	We have included how the input interacts with the convolutional layer and also introduced the concept of depth of the resulting volume. However, it is needed to explain the how the total size of this output is computed and what it depends on. There are three main parameters that play a role in this decision: \textit{depth}, \textit{stride} and the \textit{zero-padding} \cite{Karpathy2016}.
	
	\begin{itemize}
		\item DEPTH: it is a hyperparameter that corresponds to the amount of kernels used in the convolutional layer when learning from the input.  The set of neurons that extent through the depth dimension operating in the same region is called depth column. This is the previous mentioned red block shown in figure \ref{fig:mesh11}.
		\item STRIDE: this indicates the displacement of the sliding filter along the image. If it has a value of 1 (non-strided), this means that the kernel will shift a position of one pixel along the width. When it is set to 2, then it moves two pixels. It rarely has values greater than 2. The moving process consists on the filter going from right to left with a hop determined by the stride value. When the right margin is reached, then it jumps to the beginning of the next row from the left margin and repeats the process in the same way until the image is completed \cite{Saha2018}. The bigger the stride, the smaller the size of the output volume.
		\item ZERO-PADDING: this is also called just \textit{padding}. It consists on adding zeros to the borders of the image. What is a hyperparameter in design is the size of this padding. This allows to control the width and height of the output volume so we can keep it the same through the different layers. When it is indicate a "valid padding", it means that no padding is added and the output size will be reduced. If we indicate "same padding", then it has a value of 1 and means that the resulting volume will have the size of the input \cite{Keras}.
	\end{itemize}

	As shown below, the total size of the obtained volume, $D$, can be computed as function of the input size, $W$, the spatial 2\acrshort{d} dimensions of the convolutional filter, $F$, the stride of the filters shifting, $S$,  and the amount of padding in the zero-padding, $P$ \cite{Karpathy2016}. 
	
	\[ D = \frac{W - F + 2P}{S} + 1 \]
	
	The convolutional layers can be grouped all together, one after another, but they are usually interpolated by what it has been previously defined as \textit{pooling} layer. This habit has the objective of reducing the the width and the height of the resulting volumes in a progressive manner in order to decrease the total number of training parameters and control the computational cost, and, therefore to avoid overfitting \cite{Karpathy2016} \todo{Include definition of overfitting or just cite it}. There can be performed two types of pooling operation: the max-pooling or the average-pooling. The former just returns the maximum value from the portion of the image where the filter is placed. The latter, computes the average of all the values in this portion \cite{Saha2018}. The pooling layer acts in an independent way on every input depth level. 
	
	Two hyperparameters are needed in order to configure the spatial portion in which the pooling is computed: the  filter size, related to length of height and width, and the stride. However, for this operation zero parameters are introduce since it consists on a fixed operation over the input data. In figure \ref{fig:mesh12}, it is shown how the max-pooling operation is performed \cite{Karpathy2016}. 
	
	\begin{figure}[ht]
		\centering
		\captionsetup{justification=centering}
		\includegraphics[scale=0.7]{max-pooling}
		\caption{Max-pooling operation with a filter size of 2 and a stride of 2 for a single depth filter. Each colour represents a the action portion for each operation \cite{Karpathy2016}}
		\label{fig:mesh12}
	\end{figure}
	
	Together a convolutional layer and a pooling layer constitute a complete layer structure typical in a \acrlong{convnet}. The number of these complete layers can vary depending on the design and the task needs. These represent the main core of the network before passing to the last layers and calculating the final output \cite{Saha2018}.
	
	For the final step, a common way of acting is to apply some non-linear activations in order to learn from the high-level features resulted from the output volume of the last convolutional layer. This takes place in the \acrlong{fc} layers ending the model. This part has the shape of a Fully-Connected layer, as the one described in the basic model of \acrshort{ann}. To make the weights from the convolutional layers available to feed this last part of the model, it is necessary to flatten the output volume and obtain a column vector. The learning process is possible due to a backpropagation operation performed in every epoch of training. As a total output, the values that represent the classification of the different samples into one or another category are obtained \cite{Saha2018}. A common classification technique is the one called \textit{Softmax}. This normalizes the result of the previous \acrshort{fc} layer into a vector whose values follow a distribution that total sum results in 1. This type of output is called \textit{soft predictions} and represent the probability for each sample of belonging to a category in the classification \cite{Mahmood2018}. However, there exists other activation functions that can compute the classification output in other ways. 
	
	In figure \ref{fig:mesh13} an scheme of a common \acrlong{cnn} is shown. This takes an input image, compute the features along three groups of convolutional layer plus max-pooling and, finally, includes three \acrlong{fc} layers, being the last one a softmax function layer with as many neurons as classes that gives the soft predictions for this image of belonging to each class.
	
	\begin{figure}[ht]
		\centering
		\captionsetup{justification=centering}
		\includegraphics[scale=0.4]{cnn}
		\caption{Representation of a typical CNN model \cite{Hinz2016}}
		\label{fig:mesh13}
	\end{figure}

	\acrfull{cnn} were originally designed to work with images but they have been applied to other fields as audio. An image can be defined as a matrix of values, i.e. pixels, in two or three dimensions, so the only prerequisite to use a \acrshort{cnn} is to have the input data in this form. For audio researching, it has been commonly used the log-scaled mel spectrogram, that is a way of representing audio in a scale different than frequency that is more similar to humans perception. It has been used in plenty of works with \acrshort{cnn} and also combined with other techniques \cite{Salamon2017} \cite{Piczak2015} \cite{Kumar2017}.
	
	
	

	

	
	
	
	
	
	 
	
\subsection{\acrfull{vgg} model}
\label{subsection:vgg}
	
	\acrshort{cnn} are usually able to achieve really good results and even improve human skills on \acrlong{cv} tasks, for example, on recognizing object in an image. With the exponential growth of the researching works on this field, some challenges have appeared so as to promote the creation of new systems and test their efficiency and results. This is the case of the \acrfull{ilsvrc}, based on the database of the same name, ImageNet. As a solution for the proposed exercise, the investigators from the \acrfull{vgg} in the University of Oxford implemented a new system achieving the first position and winning the challenge in 2014 \cite{ImageNet2014}. The work they proposed consists of a study of the depth in a \acrlong{convnet} architecture and how this can affect to the accuracy on the goal of large-scale image recognition \cite{Simonyan2015}. To try this, it was necessary to increase the number of layers in the network, which was viable due to use a small size of convolutional filters in all of them.
	
	For the training step of their system, they used an input image with standard size of $224 \times 224$ in \acrshort{rgb} format. The principles to build the architecture are detailed below:
	
	\begin{itemize}
		\item The input image crosses a bunch of convolutional layers in which the kernel has a size of $3 \times 3$.
		\item The convolutional stride has a value of 1 pixel.
		\item The padding is fixed to 1 pixel, so the dimensions of the input do not change during the convolution.
		\item Max-pooling is also included with a window size of $2 \times 2$ and a stride value of 2 pixels.
		\item Two \acrfull{fc} layers with 4096 channels after all the \acrlong{convnet} layers.
		\item One \acrshort{fc} layer with 1000 channels to perform the \acrshort{ilsvrc} classification.
		\item A soft-max layer is employed for the final layer.
		\item All hidden \acrlong{convnet} layers use the non-linear function \acrshort{relu} as activation function.
	\end{itemize}

	All the designs that the creators came up with are based on these initial guidelines, except from just one case where \acrfull{lrn} is applied, previously mentioned in subsection \ref{subsection:ann-cnn}. They just differ from each other on the number of layers, starting with 11 the first approach and ending with 19 the last one. The different architectures are specified in the table \ref{table:3} \cite{Simonyan2015} and are ordered from A to E.
	
	% Table configuration of VGG
	\begin{table}[t!]
		\begin{center}
		\captionsetup{justification=centering}
		\includegraphics[scale=0.4]{vgg-confs}
		\caption{VGG ConvNet configurations.}
		\label{table:3}
		\end{center}
	\end{table}

\subsection{\acrshort{vgg}ish model}
\label{subsection:vggish}
	
	The model we used in our project for feature extraction presents a configuration with principles similar to the ones explained in the previous subsection \ref{subsection:vgg} but with slight changes that their developers have included in order to adapt it to the audio approach. As previously mentioned in subsection \ref{subsection:data-access}, the implementation of this model has been obtained from the website of the database Audio Set, in which a code repository is public available \cite{Ellis2017} and all the information about the implementation is included.
		
	The architecture is based in the configuration A from table \ref{table:3} with 11 weights. The differences respect to the original network are listed below: 
	\begin{itemize}
		\item The input size was changed from $224 \times 224$ to $96 \times 64$ because of the log-mel spectrogram audio inputs.
		\item They built the implementation with just four groups of convolutional and max-pool layers, instead of five.
		\item For the last \acrshort{fc} layer, they decided to build it just with 128 channels since it is the layer that compacts the final embedding and defines its dimensions.
		\item Also the final \textit{Softmax} layer is not used.
	\end{itemize}
	Figure \ref{fig:mesh2} shows how the configuration of the final \acrshort{vgg}ish model looks.
	
	\begin{figure}
		\centering
		\captionsetup{justification=centering}
		\includegraphics[scale=0.25]{model-vggish}
		\caption{VGGish architecture}
		\label{fig:mesh2}
	\end{figure}

	\subsubsection*{Input stage}
	
	Before passing the data through the whole \acrshort{cnn}, a preprocessing stage has been included by the developers in which the input audio will suffer some transformations. 
	
	\begin{enumerate}
	\item In the first place, after loading the input audio, the sample rate and the number of channels are checked to be 16 kHz and monochannel, otherwise the file is transformed to satisfy these conditions.
	\item When the data is ready, the next step is computing the spectrogram. For this, it is necessary to calculate first the \acrfull{stft} of the signal. The operation is performed by using a sliding Hanning window of 25 ms with a hop size of 10 ms. As a result, just the magnitudes that correspond to the positive frequencies are retained, since the ones corresponding to the negative part of the spectrogram are redundant.  \todo{CPM: a lo largo de la memoria se habla un montón del espectrograma así que parece bueno que haya un apéndice describiéndolo.}
	\item  Transforming the spectrogram to Mel scale is what follows. To do so, they compute all mel frequency bins that are going to modify the values of the spectrogram in frequency domain by following a hertz to mel conversion formula \todo{citation? CPM: Yes, please. Estoy sin red ahora mismo, ya me dirás si no la encuentras.}. The result is a mel spectrogram from 125 Hz to 7,500 Hz divided in 64 bins.
	\item Then, the log-mel spectrogram is calculated by doing the logarithm of the previous result plus a small offset value to avoid the logarithm of $0$.
	\item As a final step, they compute a framing operation over the log-mel spectrogram. The resulting are non-overlapping examples of 0.96 seconds, in which 64 mel bands and 96 frames are contained, each frame with a duration of 10 ms.
	\end{enumerate}

	Therefore, the result obtained is an ensemble of 10 frames, approximately one per second, each of them with size $96 \times 64$, i.e, 96 frames and 64 mel bands.
	
	\subsubsection*{Embedding stage}
	
	Once the initial processing part is done and the log-mel spectrogram matrix is computed and divided into the desired number of frames, it is used as input data for the \acrshort{vgg}ish \acrshort{cnn}. After all the computations inside the network, each example is converted into an embedding of size 128 giving a result of one of these per second of the original audio file.
	
	\subsubsection*{Postprocessing stage}
	
	As final step, they performed some post-processing of the resulting embeddings. A \acrfull{pca} \cite{Abdi2010} transformation is done joint with a whitening process. Also, a quantization to 8 bit for each embedding element. All these actions are computed with the purpose of making the final output compatible with the embeddings obtained from the YouTube-8M database.
	
	\todo{Transfer learning and why choosing this type}
	
\subsection{Why \acrshort{vgg}ish}

	For our goal, one of the toughest tasks consisted on the selection of data that properly adapted to our problem and its preparation so as to obtain features that allowed us to characterize every acoustic event from a violent point of view. Since our first efforts of finding an available dataset containing enough violent scenes was driving us to a dead end, we decided to take advantage a huge database which let us rethink the standpoint about how we were going to address the problem. As it was mentioned in subsection \ref{subsection:our-point-of-view}, one of the main questions was how to define the term violence for each victim depending on how her certain situation. After finding the Audio Set database, previously explained in section \ref{section:audioset}, with all its variety of samples, we had a wide range of audio data to work with. This is how we came up with the system explained below in \ref{subsection:violent-classes}.
	
	At this point, not only we had an idea but we had already found a data resource to start with. However, the issue was related to what kind of features could be extracted in order to categorize events from different nature with a unique violent label. An acceptable conception of the term violence could be expected to cover all kinds of short high gain \todo{CPM: no sé qué quieres decir con short high gain} events such as hits, smashes, gunshots, yells, etc. Also, we would like to introduce sounds that were likely to happen in a domestic environment within a \doubt{tense} atmosphere as children crying, dog barking or glass breaking. However, we also wanted to take into account the possibility of including other cases not usually consider violent a priori. For example, the sound of keys jangling or the noise produces by a shaver machine. These situations may be too particular and just would be present in few uses, but this is how we understand the problem given the advice of the social sciences experts in the UC3M4Safety team.
	
	So, our first intention was to apply some audio processing techniques to extract low-level features, as the ones previously explained in \ref{subsection:features-and-methods}. Even though there are plenty of previous works and a lot of tools to work in this way, it was not sure which path should we take in order to decide what features better fitted our task. Apart from this, since the database had such a big volume it would have supposed an enormous cost of time to compute features every time we wanted to try new type of categories. Moving on, by following the advances on finding new level features already mentioned in \ref{subsection:features-and-methods}, we decided to investigate new methods of extraction based on the use of \acrlong{nn} models. Nevertheless, even thought the features obtained in this case had been more appropriated, the time consumption of training a big model was one of the aspects that did not totally convince us.
	
	The previous selection of Audio Set as our dataset allowed us to get to know the \acrshort{vgg}ish model proposed by Google researchers for feature extraction. This system loads the parameters already learnt from training with another huge dataset as YouTube-8M. This is possible due to apply transfer learning idea, explained above in subsection \ref{subsection:transfer-learning}, that consists of leveraging features or weights extracted from certain models and use them in simpler ways for different tasks \cite{Sarkar2018}, so all the computational cost and training time is not a problem any longer. Finally, we decided to put in practice this pre trained embedding extractor by loading the given parameters so as to obtain our final input representations.
	
\section{Our approach}
\label{section:our-approach}

	In order to start describing our approach, we will first explained in \ref{subsection:input-data} how we obtained the input data to work with by using the resources previously explained in \ref{section:audioset} and \ref{section:feature-extractor}, and now we will move to the implementation of the whole model in section \ref{}. \todo{CPM: falta la ref a la sección.}
	
\subsection{Input data}
\label{subsection:input-data}

	Different phases took place when trying to obtain all the necessary data from the YouTube videos specified in the database. We will explain them from the first step of deciding which classes better fit our problem to the last part in which the desired embeddings are achieved.
	
\subsubsection{Violent classes}
\label{subsection:violent-classes}

	In subsection \ref{subsection:our-point-of-view} we mentioned our idea about giving the victim a choice of defining her own perception of violence, so the final implementation can adapt to her situation in a better way. To do so, we have taken advantage of the ontology provided by the Audio Set creators we explained in subsection \ref{subsecition:ontology}. 
	
	Our system has been implemented based on the idea of using the parent-children relationship among the different nodes to ease the process of selecting among the 632 classes. It must go through all the branches so as to offer the victim the possibility of choosing any of the audio event categories. However, instead of consider each label individually, this starts the way from the parent classes down to the children ones.
	
	% Flowchart of user interact system
	\begin{wrapfigure}{R}{0.4\textwidth}
		\centering
		\captionsetup{justification=centering}
		\includegraphics[scale=0.4]{select-vio-cls}
		\caption{Flowchart about selecting violent classes}
		\label{fig:mesh3}
	\end{wrapfigure}
	
	
	Let's say we begin from the class "Human sounds" that is the top level of all sounds emitted by humans contained in the dataset. The system will ask the user if she wants to advance in that direction, i.e., to go across the branches that belong to that part of the tree. If the answer is positive, it would go for the next class, that in this case it would be "Human Speech". It will advance this way until there were no children nodes in the actual class. When this happens, the user will be asked to add the label to the record of \textit{violent classes}, that will be saved in a text file so they can be read by other parts of the model further on. If the user does not want to go deeper, she will be asked to add the current class to the record. If the answer is "No", then they system will jump to the next sibling category. The corresponding flowchart is shown in figure \ref{fig:mesh3}.
	
	This way of flowing through the different classes allows to skip those parts that are not related to our problem. For example, as we can see in figure \ref{fig:mesh1}, one of the top parent labels is "Natural sounds", that relates to sounds from weather phenomena. In most of the cases these classes will not be selected so the whole branch can be skipped. 
	
\subsubsection{Downloading videos}

	The following step consists in achieving audio files that belong to the chosen labels. For this purpose, we have made use of the \acrshort{csv} files that were explained in subsection \ref{subsection:data-access}. For each included video, we took its ID and build the corresponding YouTube \acrshort{url}. Once downloaded, we trimmed the file considering the onset and offset and, finally, converted it to audio format (\acrshort{wav}). In our script, we can pass as a parameter the identifier of the desired classes in comma-separated format and either the number of videos per class for a balanced set or a total number of downloads for an unbalanced set. However, there might be some errors when obtaining all the data: the two most common cases are due to lack of enough videos of the desired type in the dataset or because the video is not available anymore on YouTube. When this happens, a message is shown to the user.
	
	It it also worth mentioning that throughout the developing of this downloading task, a script has been coded to achieve the whole dataset in both formats, video and audio, for future works. We are not going to specify anything else since this feature was not finally used.
	
\subsection{Extracting embeddings}
\label{subsection:extracting-embeddings}

	At this point, all the desired data has been already downloaded to extract the embedded features that will be used to train the model. For this part, we have used the \acrshort{vgg}ish network explained in subsection \ref{subsection:vggish}. Since the audio files duration is usually 10 s, and the embedding extractor gives as a result a vector of size $1 \times 128$ for each second, we will obtain a $10 \times 128$ feature matrix composed by values within the range 0 - 255. Therefore, our input feature matrix will have a size of $(number of audios) \times 10 \times 128$. The corresponding labels will be stored in a vector of size $(number of samples)$.
	
	There are some points about the data obtained in this step that should be commented. One of them is about the length of the audio files. As it was previously indicated in \ref{subsection:data}, most of them last 10 s because the creators decided to set this duration for the audio events, but this can change if the video is originally shorter. For these situations, since the model is configured to have an input of $10 \times 128$, the embedding matrix of the shorter clip must be padded with zero-rows to achieve the required dimensions. Even though this does not happen very often, there might be some silent segments that will be labelled with the category of the rest of the audio. \todo{Reference from cases in which the matrix is 2D}
	
	The other case is related to what was explained in \ref{subsection:data-access}. For the recent explanation of how to extract the input features from the audio files we have utilized the first manner of accessing the data, i.e., by reading the text files with the videos information before downloading. There is a second option of using the already extracted embedding features. However, these do not look exactly the same when comparing them to the ones obtained from our own extractor. This difference is due to the implementation of the given code differs from their internal production system in computing issues such as underlying libraries in the installation of \acrshort{vgg}ish and hardware \doubt{equipment}. In spite of this, the result in classification tasks are expected to be equivalent. In order to prove this assumption, we decided to try a small system with both kinds of data. The reason for this check is that although the development of the methodologies could be made by using the first and faster manner of accessing the data, for the real implementation in the pendant device within the EMPATIA project (out of the scope of this project) only the second way will be feasible.
	
	\todo{Include json format of tfrecord files}
	
\subsection{Assessing the differences between the two types of data access}
\label{subsection:exploring-differences-between-two-types-of-data-access}

	In order to check what is explained above we have decided to run a little experiment in which a toy classification is performed. Also, we wanted to visualize the different features to check if we could appreciate common patterns by using the \acrfull{tsne} algorithm, which will be explained later in this section.
	
	Our first step consisted in determining a subset extracted from the original dataset. We thought about choosing for this small application a subset composed by three classes that could be considered violent and other three that were non-violent. Apart from this, we paid attention to the number of samples per category to pick some classes over others. Finally, we ended up picking up the labels detailed in table \ref{table:4} and a number of 80 samples for each of them, which led us to a total of 480.
	
	% Table for chosen classes
	\begin{table}[h]
	\begin{center}
		\begin{tabular}{||m{7em} | m{23em} ||}
			\hline
			\textbf{Class} & \textbf{Description} \\
			\hline\hline
			Screaming & A sharp, high-pitched human vocalization; often an instinctive action indicating fear, pain, surprise, joy, anger, etc. Verbal content is absent or overwhelmed, unlike Shout and Yell. \\
			\hline
			Crying, sobbing & Sound associated with the shedding of tears in response to an emotional state, arising from slow but erratic inhalation, occasional instances of breath holding and muscular tremor. \\
			\hline
			Gunshot, gunfire & The sound of the discharge of a firearm, or multiple such discharges. \\
			\hline
			Animal & All sound produced by the bodies and actions of non-human animals. \\
			\hline
			Engine & The sound of a machine designed to produce mechanical energy. Combustion engines burn a fuel to create heat, which then creates a force. Electric motors convert electrical energy into mechanical motion. Other classes of engines include pneumatic motors and clockwork motors. \\
			\hline
			Printer & Sounds of a computer peripheral which makes a persistent human readable representation of graphics or text on paper or similar physical media. \\
			\hline
		\end{tabular}
	\end{center}
	\caption{Chosen classes for a small classification. \textit{Screaming}, \textit{Crying, sobbing} and \textit{Gunshot, gunfire} are considered as the violent ones.}
	\label{table:4}
	\end{table}
	
	For the classification task, we decided to create a small \acrshort{cnn} composed by few layers. Since our input data are matrices of shape $10 \times 128$, these were treated as images so the model was built with layers that perform spatial convolution \cite{Levoy2012}. The architecture is detailed in figure \ref{fig:mesh5}. We used a small kernel size of $3 \times 3$, zero padding so as not to the shape of the output and a activation of function \acrshort{relu}. Two dense\footnote{Dense is a also a common form to refer to \acrlong{fc} layers. Since each neuron is fed with an input from all the neurons in the previous layer we can say that they are densely connected \cite{Rampurawala2019}} layers are added at the end, first one with also \acrshort{relu} as activation function and the second one with \textit{softmax} to perform the classification, and as many filters as the number of classes.
	
	In order to measure the results, since our subset is balanced, we could have evaluated our model by computing the accuracy and the confusion matrix. More information about these metrics can be found in the appendix \ref{appendix:metrics}. In figure \ref{fig:mesh7}, the four confusion matrices corresponding to the training and test phases for both types of data are shown. Also, in table \ref{table:5} are included the accuracy for each case. The results are more accurate when training with the embeddings extracted directly from the audio files we downloaded. When obtaining them from the \acrlong{tf} files, they value of the metrics indicate a worse performance. However, we opted for this second manner because the performance difference is not too big and it requires  much less computational cost and time loss.
	
	% Four confussion matrices
	\begin{figure}[htb]
		% Whole figure
		\captionsetup{justification=centering}
		\begin{subfigure}[b]{\textwidth}
			% Start with figure wav
			\centering
			\captionsetup{justification=centering}
			\includegraphics[width=0.5\linewidth]{wav_train}%
			\includegraphics[width=0.5\linewidth]{wav_test}
			\subcaption{Confusion matrices when embeddings are extracted from audio files}
		\end{subfigure}
		\vskip\baselineskip
		% Start with figure tfrecord
		\begin{subfigure}[b]{\textwidth}
			\centering
			\captionsetup{justification=centering}
			\includegraphics[width=0.5\linewidth]{tfrecord_train}%
			\includegraphics[width=0.5\linewidth]{tfrecord_test}
			\subcaption{Confusion matrices when embeddings are taken from .\textit{tfrecord} files}
		\end{subfigure}
		\caption{Confusion matrices}
		\label{fig:mesh7}
	\end{figure}

	% Accuracy values train and test
	\begin{table}[h]
	\begin{center}
		\centering
		\begin{tabular}{|| m{7em} | m{7em} | m{7em} ||}
			\hline
			format / subset & \textbf{Train} & \textbf{Test} \\
			\hline\hline
			\textbf{Audio file} & 0.98 & 0.63 \\
			\hline
			\textbf{.\textit{tfrecord}} & 1.0 & 0.72 \\
			\hline
		\end{tabular}
	\end{center}
	\caption{Accuracy values for audio and .\textit{tfrecord} files}
	\label{table:5}
	\end{table}

	\doubt{For the training phases, it can be appreciated that there may be an over-fitting since the accuracy is perfect. This means that the \acrshort{nn} stop improving its capacity of learning how to solve the problem in a certain moment of the training task. Instead, it does learn some behaviour pattern that the training data follows. This impacts negatively in the model since the new data that the system will have to learn from will look different and will not follow these same rules \cite{Jabbar2015}. CPM: no le daría más importancia a esto.}. In spite of this result, we did not give it so much importance since we just wanted to do a quick check of the similarity between the two types of data which can be appreciated due to the similarity of both metrics results.
	
\subsubsection{Dimensionality reduction for visualization}
\label{subsection:dimensionality-reduction-for-visualization}
	
	Apart from the classification exercise, we wanted to see if by plotting the data samples we were able to identify or appreciate some common patterns. In our problem, each of our samples is characterized by a matrix of features with 128 columns, which means that we were working with data belonging to a high-dimensional space. Visualizing this type of data has always been a case of study for many different fields. Plenty of methods have been published so as to find a solution for this task. Some of the most accepted methods consists on reducing the dimensionality of the data so this can be transformed from the high-dimensional space to a lower one and can be visualize in a common scatter plot of 2D or 3D. In particular, these techniques rely on the idea that within a multivariate sample denoted as $x_i = [x_{i1},..., x_{in}]^T$ and considered to be a point that corresponds to a $n-dimensionality$ space, a $d-dimensionality$ space can be found, so that $d < n$. %, in which the data \doubt{is included}. 
	If this is possible, then the observations can be transformed to this lower dimensional space $d$ without any loses \cite{Kaski2011}. 
	
		% Architecture of model of user interact system
	\begin{wrapfigure}{R}{0.4\textwidth}
		\centering
		\captionsetup{justification=centering}
		\includegraphics[scale=0.6]{try-wav-tf}
		\caption{Architecture to see how the different embeddings work}
		\label{fig:mesh5}
	\end{wrapfigure}
	
	One of the most common and well-known dimensionality reduction methods is the one known as \acrfull{pca}. This follows the idea previously explained. It specifically aims at extracting the \textit{important} information from the original data and transform them into a set formed by orthogonal variables which are actually known as principal components. This is done by multiplying the matrix data $X$ by a projection matrix $Q$ that contains the coefficients of the linear combinations that allow the conversion. The projections must be orthogonal to each other and they represent the data maximum variance in descending order, being the first component the one with largest variance \cite{Abdi2010}. In fact, each of the projections correspond to an eigenvector in descending order following the value of the eigenvalue. So, the first component will be the eigenvector with the highest eigenvalue. It has been proved to be one of the most reduction dimensionality techniques nowadays. Its use is completely accepted and it is implemented in many famous software libraries. However, it presents some limitations. One of them consists on just considering linear combinations of the original data. When the relation is non-linear, a dimensionality reduction with this technique may result in a loss of information \cite{AmatRodrigo2017}. 
	
	When the relation between the different subspaces cannot defined as linear, there have been developed other methods with such as \acrfull{tsne}. This algorithm appears as an extension of the previously developed \acrfull{sne} \cite{Hinton2003}. Both are based on the same idea of a new way of measuring the similarity between samples. Instead of comparing two observations, let's call them $x_i$ and $y_j$, by computing the euclidean between them, this is done by calculating the conditional probability $p_{j|i}$ of $x_j$ being picked as a neighbour of $x_i$ considering that the samples belong to a Gaussian distribution centered at $x_i$. Its depends on how far the samples are from each other, i.e., it is high when they are close and minimum when there are totally separated \cite{VanDerMaaten2008}. Apart from this, two analogous observations are created in the subspace of lower-dimensionality, $y_i$ and $y_j$, and conditional probability $q_{j|i}$ is computed in this situation. It is important that, for $y_i$ and $y_j$ to be faithful representations of $x_i$ and $x_j$, both conditional probabilities must be equal.
	
	In order to calculate the probabilities, a crucial factor is the variance of the Gaussian distribution. There is not a unique valid choice for this parameter, so \acrshort{tsne} performs a binary search so as to find the optimal one \cite{AmatRodrigo2017}. This is also influenced by what is called the perplexity. This can be defined as an assumption of the number of adjacent neighbours for each point. It is a value that is fixed by the user, but it usually belong within a range from 5 to 50 \cite{VanDerMaaten2008}. 
	
	There are several considerations that should be known before looking at a representation of data from this algorithm \cite{Wattenberg2016}. Actually, it is not an easy task to understand this kind of plots, since the distance between points in the new subspace are not related to the real euclidean distance, which has been denoted as "The Crowding Problem" \cite{VanDerMaaten2008}. This means that the groups cannot be interpreted as real collections of data in the the original dimension. However, in order not to misunderstand the data distribution, a couple of visualizations varying the parameters usually tends to be done so that the conclusions can be based on more than one result. In figure \ref{fig:mesh8} we show ten \acrshort{tsne} outputs, five for each type of data for diverse values of perplexity from 2 to 50.
	
	% Four confussion matrices
	\begin{figure}[bt]
		% Whole figure
		\captionsetup{justification=centering}
		\begin{subfigure}[b]{\textwidth}
			% Start with figure wav
			\centering
			\captionsetup{justification=centering}
			\includegraphics[width=0.2\linewidth]{wav-perp-2}%
			\includegraphics[width=0.2\linewidth]{wav-perp-5}%
			\includegraphics[width=0.2\linewidth]{wav-perp-20}%
			\includegraphics[width=0.2\linewidth]{wav-perp-30}%
			\includegraphics[width=0.2\linewidth]{wav-perp-50}
			\subcaption{From audio files}
		\end{subfigure}
		\vskip\baselineskip
		% Start with figure tfrecord
		\begin{subfigure}[b]{\textwidth}
			\centering
			\captionsetup{justification=centering}
			\includegraphics[width=0.2\linewidth]{TF-perp-2}%
			\includegraphics[width=0.2\linewidth]{TF-perp-5}%
			\includegraphics[width=0.2\linewidth]{TF-perp-20}%
			\includegraphics[width=0.2\linewidth]{TF-perp-30}%
			\includegraphics[width=0.2\linewidth]{TF-perp-50}
			\subcaption{From .\textit{tfrecord} files}
		\end{subfigure}
		
		\caption{t-SNE results from both formats with a legend that shows the labels of the data in the original 128D space}
		\label{fig:mesh8}
	\end{figure}
	
	\todo{CPM: aumentar las figuras para que se puedan leer las leyendas.}
	
	The selection of the perplexity value depends on the number of observations per class \cite{Wattenberg2016}. Since our subset has 80 samples for each category, we can consider that a proper value is 20 or 30. As we can see, a boundary cannot be extracted among the different labels, but we can see some grouping patterns in the data that correspond to the original labelling and helps us to confirm the similarity of the two different given types. It is true that this method should never be used as an algorithm for clustering itself, but it is a good resource as a backing strategy to other results, as in this case. 
	
\section{Methodology}
\label{section:models}

	% Explain the methods used:
	% 1) Data augmentation and SMOTE
	% 2) SVM 
	% 3) Explanation of the learning process in deep learning (SGD, epochs and batch size)
	% 4) CNN
	% 5) LSTM

	% !TeX spellcheck = en_GB

\subsection{\acrfull{smote}}
\label{subsection:smote}

	\acrshort{smote} consists of an over-sampling method whose function is to generate synthetic samples in order to populate the minority class in a classification problem by making use of the proper observations that are already contained in the data \cite{Chawla2002}. 
	
	Essentially, this technique works by choosing samples from the less populated class that are represented in the features space, placing a line between two of these samples and then select a point inside the line at a random position. First, an arbitrary sample is picked and then a number of \textit{k} closest neighbours in the feature space are selected from the same class. One of the neighbours are randomly chosen and a synthetic sample is generated \cite{Browniee2020}. The number of neighbours \textit{k} can vary depending on the amount of over-sampling needed. The standard value which is the one proposed in the publication paper is \textit{k = 5} \cite{Chawla2002}. 
	
	For the generation of the synthetic samples, the distance between the feature vector first arbitrarily selected and one of the nearest neighbours is computed. Then, the resulting value is scaled by multiplying it by a random number that lies in the uniform distribution $U(0,1)$. Finally, this new sample, \textit{S}, is added to the initial feature vector and placed in between these two observations in the feature space \cite{Chawla2002}. \textit{S} can be defined as follows:
	\[S = x + u\cdot(x^{R} - x) \]
	where \textit{x} is the arbitrary selected sample the first time, \textit{u} is the random value from $U(0,1)$ and \textit{x\textsuperscript{R}} is one of the closest neighbours \cite{Blagus2013}. In figure \ref{fig:mesh47}, it is shown a graphical representation of the algorithm. In this case, \textit{X} is the original sample for which the closest neighbours are represented as $X_{i}$ where $i=1,2,3,4,5$. The generated samples are $Y_{1}$ and $Y_{2}$ that are placed in the line segments.
	
	\begin{figure}
		\centering
		\captionsetup{justification=centering}
		\includegraphics[scale=0.3]{SMOTE}
		\caption{SMOTE algorithm \cite{Xie2015}}
		\label{fig:mesh47}
	\end{figure}

	The common way of applying this technique is in the train set before performing the fit of the model. Usually, an under sampling is performed in the most populated classes and, then, the data augmentation is done for the minority ones \cite{Browniee2020}. Our implementation follows this way of acting and it is explained in section \ref{section:input-data-preparation}.
	
	 
	
	% !TeX spellcheck = en_GB

\subsection{\acrfull{svm}}
\label{subsection:svm}

	\todo{Adapt...}
	Considering instances of data with their respective ground-truth labels, this classifier finds the frontier defined as a hyperplane that maximizes the distance between the two observations closest to each other that belong to different classes and are more likely to be misclassified. These two points are called support vectors \cite{Fu2011}. This type of method is characterized for being a binary classifier. However, its implementation has been transported to multiclass problems by combining various \acrshort{svm}s in order to stablish a decision criterion so the whole system can chose among $Q$ categories. This can be done by following either one-versus-all or one-versus-one approach. The former addresses the problem training $Q$ classifiers so as to differentiate between data from one class and data from the other $Q-1$ classes. The latter treats the problem by training $Q(Q-1)/2$ \acrshort{svm}s to distinguish among all possible combinations of categories. In any of the approaches, the classification of a certain observation is performed by computing the distance between the sample data and the hyperplane that defines the frontier \cite{Barchiesi2015}. The one-versus-one is the one implemented by \acrfull{svc} from Scikit-learn library and is the one we will use in one of the experiments explained in section \ref{}. In figure \ref{fig:mesh10} it is shown a representation of the decision boundaries that could designed by other type of methods against the hyperplane a \acrshort{svm} classifier would draw.
	
	\begin{figure}[h]
		\centering
		\captionsetup{justification=centering}
		\includegraphics[scale=0.2]{svm}
		\caption{Visualization of a hyperplane set by \acrshort{svm} and other decision boundaries \cite{Drakos2018}}
		\label{fig:mesh10}
	\end{figure}
	
\todo{Add requisites for sklearn algorithm and details about implementation in our model}
\todo{Include definition of kernel and cases}
	
	% !TeX spellcheck = en_GB

\subsection{\acrlong{rnn} and \acrlong{lstm}}

	Human learning does not happen at each moment independently in the sense that every time something new is learned, it depends on the previous knowledge in order to interpret it. This can also be explained by saying that our understanding is persistence.
	
	Traditionally, \acrlong{ann} are not able to act in this way. They \doubt{cannot use the time} as a property to infer conclusions or predictions from the previous instances that belong to a sequence of data. To address this task \acrfull{rnn} have been developed, so they can take information and make it persistent. In figure \ref{fig:mesh39}, it is shown a block of a \acrshort{rnn} in which the network, \textit{A}, is fed with an input \textit{x\textsubscript{t}} that is actually a sequence of data. The output is the value \textit{h\textsubscript{t}}. An inner loop takes the information from the outcome and pass to the input for the next learning step. This can be easily seen in the unrolled part of the right, how the information flows from one element to the next one \cite{Olah2015}. 
	
	\begin{figure}[h]
		\centering
		\captionsetup{justification=centering}
		\includegraphics[scale=0.35]{rnn}
		\caption{Scheme or a recurrent neural network about how the loop works }
		\label{fig:mesh39}
	\end{figure}

	This type of networks has became the standard when dealing with sequential data. They have been applied in many tasks, such as language modelling, speech recognition, translation, etc. However, the most simple approach present a problem when the information that must be considering when long-term dependencies. For example, if the task consists of predicting a new instance based on the previous ones in a not really long sequence in which the dependency resides on close instances, normal \acrshort{rnn} can be used and work in the way explained above. When the distance inside the sequence between the predicted element and the ones with the important information that this new one depends on is too large, the network cannot learn this connection \cite{Olah2015}. 
	
	\begin{figure}[h]
		\centering
		\captionsetup{justification=centering}
		\includegraphics[scale=0.35]{long-term-dependencies}
		\caption{Example of long-term dependencies situation. If the predicted output h\textsubscript{n-1} depends on x\textsubscript{0} and x\textsubscript{1}, the vanishing gradient problem may appear.}
		\label{fig:mesh40}
	\end{figure}
	
	This happens due to the value of the gradients becomes very small during the backpropagation in the learning process. Once the loss function for the new predicted output is calculated, this must be propagated through the rest of the network in order to update the weights. In a typical neural network model, \doubt{the neurons that get updated are those from the hidden layer directly previous to the output one}, but, since the data is a sequence, the neurons from all the earlier layers must be updated as well. The actual problem appears when renovating the value of the weights through time, i.e. updating the weights used to connect the hidden layers in the unrolled temporal loop. So, if the value of the gradient becomes very small the updating process start to be null when finding the new values of the weights and the model stop learning. This is known as the vanishing gradient problem \cite{SuperDataScienceTeam2018}.
	
	As a solution, a new type of \acrshort{rnn} was developed and named as \acrfull{lstm} networks. These can be defined as a recurrent network that can learn long-term dependencies, so the vanishing gradient problem is not an issue for these models. Nowadays, they are widely used in several fields, since they can remember information along long periods \cite{Olah2015}. The main structure along is the same shown above in figure \ref{fig:mesh39}. The difference from the way of working in a common \acrshort{rnn} is the inner architecture in the neural network. In \acrshort{lstm}, the \textit{A} modules are defined as shown in figure \ref{fig:mesh41}. The notation used is as follows: the rectangles denotes a neural network layer, the circle shapes refer to point-wise operations and the arrows means vector transferring.
	
	\begin{figure}
		\centering
		\captionsetup{justification=centering}
		\includegraphics[scale=0.29]{lstm-module}
		\caption{LSTM module with the representation of the different operations that take place inside of it}
		\label{fig:mesh41}
	\end{figure}

	The main concept of a \acrshort{lstm} network resides on its cell state and the different gates. The cell state, represented in the diagram by the upper arrow, is the one in charge of transferring information all the way through the chain of modules. It can be thought as the memory of the whole network. Its objective is to just carry information considered relevant for the model. This way is how the \acrshort{lstm} brings the information from the first layers to the last ones without suffering the vanishing gradient problem. The content of the cell state is modified by the point-wise operations that acts by following the gates criteria. These are in fact neural networks that has the goal of deciding which information can be treated as relevant and so added to cell state. This process can be understood as a learning/forgetting stage \cite{Nguyen2018}. 
	
	In order to understanding how this process works, we are going the interaction of the different gates in the learning stage.

	\begin{wrapfigure}{l}{0.26\textwidth}
		\centering
		\captionsetup{justification=centering}
		\includegraphics[scale=0.25]{lstm-forget-gate}
		\caption{Step 1. Forget gate}
		\label{fig:mesh42}
	\end{wrapfigure}

	The first step is performed by what is called the \textit{forget gate} layer. It must decide which information must be included in the cell state that travels through the time chain. This is way the layer contains a sigmoid function that outputs values between  0 and 1, in which  0 means dropping the information in the previous cell state and 1 implies to keep it. As shown in figure \ref{fig:mesh42}, this is done by paying attention to the hidden state which is the output of the previous module, \textit{h\textsuperscript{<t-1>}}, and the input of the current one, \textit{x\textsuperscript{<t>}} \cite{Nguyen2018}. The resulting function $f$ is defined as follows, where $W_{f}$ are the weights of the \acrshort{rnn} and $b_{f}$ the bias.
	\[
	\ f = \sigma(W_{f}[h^{t-1}, x^{t}] + b_{f})
	\]
	
	\begin{wrapfigure}{r}{0.3\textwidth}
		\centering
		\captionsetup{justification=centering}
		\includegraphics[scale=0.27]{lstm-input-gate}
		\caption{Step 2. Input gate}
		\label{fig:mesh43}
	\end{wrapfigure}
	
	Next, the decision about what information from the current input \textit{x\textsuperscript{<t>}} must be taken to the cell state is performed. The gate that computes this operation receives the name of \textit{input gate}. This part can be divided into two different steps. First, the previous hidden state,\textit{x\textsuperscript{<t-1>}} ,and the current input feed a \acrshort{rnn} with a sigmoid for the activation function, which takes the decision of \doubt{which values must be updated} by converting them into a range limited by 0 and 1. This result denotes the importance of the value, being 0 non-important and 1, important. Then, also \textit{h\textsuperscript{<t-1>}} and \textit{x\textsuperscript{<t>}} are used as input for the \acrshort{tanh} layer so the values are mapped between -1 and 1. This last activation function is included in order to \doubt{avoid the vanishing gradient problem previously mentioned, since a function whose second derivative takes more time to tend to zero is needed}. Finally, both outputs are multiplied before the updating process in the cell state \cite{Nguyen2018}. The equations for the output of each \acrshort{rnn} module are included below, where $W$ and $b$ denotes the weights and bias for each layer \cite{Olah2015}.
	\[ i = \sigma(W_{i}[h^{t-1}, x^{t}] + b_{i}) \]
	\[ C = \tanh(W_{C}[h^{t-1}, x^{t}] + b_{C}) \]
	
	In the step 3, the output of the multiplication from the input gate layer and the output $f$ of the sigmoid function from the forget gate, modify the old cell state from the previous module, \textit{c\textsuperscript{<t-1>}}, so it can be updated. For the outcome of the step 1, a point-wise multiplication is performed, $f * C^{<t-1>}$. Then, to the output of this operation is added the resulting product of the input gate, $i * C$. This two operations can be translated as a forgetting and a learning stage. First, the values the forget gate decided to remove from the cell state are actually forgotten. Then, it must learn the new values belonging to the current input also weighted by their importance, what denotes how much the state values are going to be updated \cite{Olah2015}. The new cell state expression $c^{t}$ is included below and the process is shown in figure \ref{fig:mesh44}.
	
	\begin{wrapfigure}{l}{0.4\textwidth}
		\centering
		\captionsetup{justification=centering}
		\includegraphics[width=\linewidth]{lstm-cell-state}
		\vspace{-10pt}
		\caption{Step 3. Update cell state}
		\label{fig:mesh44}
	\end{wrapfigure}
	
	
	\[c^{<t>} = f * C^{<t-1>} + i * C\]
	
				
	Finally, the last step of the module corresponds to the \textit{output gate}. This has the function of deciding how the next hidden state is going to be like. First, \textit{x\textsuperscript{<t>}} and \textit{h\textsuperscript{<t-1>}} are used as input for a layer with a sigmoid activation function and the output $o$ is obtained. Also, the cell state \textit{c\textsuperscript{<t>}} feeds a point-wise \acrshort{tanh} so its values are clipped between -1 and 1. Both outputs are multiplied. The $o$ result decides what values form the current input must be kept in the future hidden state,\textit{h\textsuperscript{<t>}}. The actual output of the module is the hidden state. It also transports the relevant information updated together with the cell state to the next network. The outputs $o$ and $h^{t}$ can be defined as shown below. Also, in figure \ref{fig:mesh45}, the stage of this last part is included.
		
	\begin{figure}{h}
		\centering
		\captionsetup{justification=centering}
		\includegraphics[scale=0.25]{lstm-output-gate}
		\caption{Step 3. Output gate}
		\label{fig:mesh45}
	\end{figure}
	
	\[ o = \sigma(W_{o}[h^{<t-1>, x^{t}}] + b_{o}) \]
	\[ h^{<t>} = o * \tanh(C) \]
	
	
	
	
	
	

	
	
	
	
	

	
	
	

	
	
	
	
	
	
	
	
	