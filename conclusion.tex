% !TeX spellcheck = en_GB

	A couple of conclusions can be extracted from the whole development of the work related to the different objectives that were initially set. Within the realization of the steps that were considered necessary to achieve the solution of the final task, several outcomes that should be taken into account were achieved.
	
	During the first labour of finding a proper database that really adjusted the problem, we came up with several solutions. Finally, Audio Set was the best choice due to its characteristics of being formed by a really massive amount of videos in different conditions. Initially, we thought it was not going to be a problem the lack of data and that we could extract all the necessary information from all these videos. However, some limitations were discovered during the whole investigation process. The unbalanced nature of the dataset, the limitation we imposed of just using monolabel data and the quality of the different labels are some examples of the problems that we found when adapting Audio Set to our problem. Also, the own quality of the videos could be a factor that may affect the resulting embedding. A more proper dataset could be a good point to improve.
	
	However, finding this database has not just supposed a data resource. The same developers that made Audio Set, released a deep neural model called \acrshort{vgg}ish which really eased the labour of extracting features. We investigated this network and checked the behaviour of the feature embeddings and see that they can really be used to perform classification tasks within this research field. Maybe, we cannot consider this transformation as a feature space in which the violent component plays a really significant role when finding a separability between violent and non-violent events, but the feature extractor should be used to obtain embeddings from videos in better quality and labeling conditions so a more accurate result could be reached.
	
	About performing violent event detection form a gender-based violence point of view, it is clear that the definition of how these events must look is fundamental. The system designed for that purpose is just a simple method to select violent classes in a more efficient way whose utility probable does not get any further from the scope of this work. Considering the well organized ontology that Audio Set is provided with, we thought that a good saving of time would be to being able to select violent categories without checking all the labels one by one. However, the idea is mainly the same that a normal application or interacting system should follow to accomplish this task, maybe with other type of classes or more specific ones.
	
	In order to implement a learning method to classify violent events, we decided to mix a multiclass classification and a binary one. This solution did not came at first. One of the initial approaches was to perform a normal binary model with data tagged as \textit{violent} or \textit{non-violent}. However, we thought it could be a good idea to focus on obtaining the multiclass results and then adapt them to the binary situation and see how this outcome is. With this purpose, we tried different methods obtaining various results, being the \acrshort{svm} model the one with a best performance on the multiclass task and the \acrshort{lstm} the one with most accurate results for the binary case. However, a total conclusion of which one is better cannot be extracted. In the multiclass classification, the output obtained generally make sense if considering the semantic of the label even thought the true positive regions were not the most populated. In the binary classes, the results can be considered accurate but the number of classes is much higher and also most of the corrected predictions belonged to non-violent events, since these are the most populated classes in the test set. The \acrshort{cnn} model did not stand out in any of the tasks, but it achieved pretty good results in the binary classification. The architecture of the neural models is an aspect that also should be taken into account, since this work did not go deep enough in that area.
