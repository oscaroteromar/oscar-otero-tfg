% !TeX spellcheck = en_GB
% INTRODUCTION text
\section{Context}

	Violence against women remains an invisible phenomenon, deeply within the victim’s private life in most cases. It is based on deep social and cultural roots and it is undoubtedly linked to unbalanced relationships between men and women in different situations and contexts, such as economics, politics and religion. In order to prevent these conflicts, the related legislation has achieved important improvements for the last years \cite{}. \todo{CPM: falta la cita.}According to the results of most of the studies, victims can be usually defined as women who endured violence during their childhood and felt socially isolated. They are also characterized for a considerable economic dependency and a low educational level \cite{}.\todo{CPM: tienes puesto que vas a introducir unas refs. Es importante porque no se pueden mencionar estudios de esta manera "most of the studies" sin citar algunos al menos}

	With the purpose of making a difference when identifying situations showing this kind of violence and applying the knowledge and technological advances acquired during this information era, machine learning and deep learning models can collect all the available data to protect eventual victims. \todo{CPM: esta frase no se entiende, ¿quién tiene el propósito de marcar una diferencia?}
	
	The main challenge is to get to know how the victim is feeling when she is under a threatening situation, for example, if she is scared or nervous, and combine this with other variables which may play an important role in the scene and might be helpful in making a decision about its characterization. There are several factors that can be considered to achieve this task. One of them is the audio, either the victim's voice or the environmental sounds.
	
	Plenty of useful information can be extracted from the acoustic scene. %of a certain place. 
	The detection of audio events can be combined to understand % is an equally good way to define 
	what is happening in a certain moment. Once these data are collected, they can be classified in different categories and thus describe several aspects of the acoustic scene. Either based on an objective definition of gender violence or on an explanation previously obtained from a particular/specific victim, this acoustic knowledge can be interpreted as dangerous for the user. \todo{CPM: no me gusta mucho la distinción entre objetiva y particular/específica porque la primera tampoco es 'objetiva'. Se me ocurre mejor 'stereotyped' and 'personalized', por ejemplo.}

	
\section{Objectives}

	The utilization of learning models to extract useful information from data has become a very common practice in most of the fields. One type of habits that have gained a lot of popularity in the scientific community is the use of multimedia data. In many cases, the samples used to train the models consist in images that belong to a certain kind of problem, such as medical imaging or object recognition. This field is known as \acrfull{cv}. Many world well known architectures and enormous data bases have been born during the study of this kind of problems.
	
	In the same way, audio data have been used to get conclusions from a lot of real world problems. In order to tackle the task of violent event detection it is important to decide what perspective is going to be adopted when defining a violent event, whether an objective point of view or a more personalized standpoint according to the victim's criteria. Apart from this, it is also necessary to extract the required features, that is, the information from the audio signals that will allow us to train the models and get the results. However, the main work will be characterized by classifying a whole scene depending on the events this is built by. Once an action sound is categorized, it can be identified as violent by checking if it belongs to the violence definition previously defined. \todo{CPM: Este párrafo es un poco confuso y contiene disquisiciones que no se entienden hasta que no se hayan hecho una serie de definiciones que supongo que se hacen más adelante. Piensa en una lista de Objetivos que empiecen con un verbo. Por ejemplo, "Encontrar (o investigar) una manera de describir una escena acústica violenta teniendo en cuenta las particularidades de la víctima." "Implementar un sistema capaz de ..."}
		
	The different acoustic scenes that may be considered for the problem can be composed by events of different nature or those that belong to just one class. This difference may cause that the techniques utilized to address the problem can differ. As a further approach, it is interesting to find a method that can distinguish among events that come from different sources of audio.
	
\section{Regulatory framework}
\todo{Tips?}
\todo{CPM: por ejemplo, asuntos relacionados con la ley de protección de datos, la ley integral de violencia de género -derechos de las víctimas y los agresores-. De hecho, yo aquí diría que el marco regulatorio de la aplicación es altamente complicado y que por eso, en el proyecto hay expertas trabajando estos temas pero que tú, en particular, para el trabajo que has desarrollado (puesto que la base de datos es de acceso público) sólo has tenido que observar las instrucciones de la bd.}

\section{Socio-economic environment}
\todo{Tips?}
\todo{CPM: aquí puedes hablar del impacto socio-económico. Lo que supone empoderar a las mujeres ...Puedes hablar del los objetivos sostenibles de las Naciones Unidas (SDG 5, gender equality) y tienes un filón.}
	
	
	
	