% !TeX spellcheck = en_GB
% INTRODUCTION text
\section{Context}
	Violence against women remains an invisible phenomenon, deeply within the victim’s private life in most cases. It is based on deep social and cultural roots and it is undoubtedly linked to unbalanced relationships between men and women in different situations and contexts, such as economics, politics and religion. In order to prevent these conflicts, the related legislation has achieved important improvements for the last years \cite{}. According to the results of most of the studies, victims can be usually defined as women who endured violence during their childhood and felt socially isolated. They are also characterized for a considerable economic dependency and a low educational level \cite{}.

	With the purpose of making a difference when identifying situations showing this kind of violence and apply all the knowledge and technological advances acquired during this information era, machine learning and deep learning models can collect all the available data to protect eventual victims.
	
	The main goal is to get to know how the victim is feeling, for example, if she is scared or nervous, and combine this with other variables which may play an important role in the scene and might be helpful in making a decision about the characterization of the ambiance. There are several factors that can be considered to achieve this task. One of them is the audio, either the victim's voice or the environmental sounds.
	
	Plenty of useful information can be extracted from the acoustic scene of a certain place. The detection of audio events is an equally good way to define what is happening in a certain moment. Once these data are collected, they can be classified in different categories and thus describe the scene. Based either on an objective definition of gender violence or in an explanation previously obtained from a particular/specific victim, this acoustic knowledge can be interpreted as dangerous for the user.

	
\section{Objectives}
	The utilization of learning models to extract useful information from the worlds data has become a very common practice in most of the fields. One type of habits that have gained a lot of popularity in the scientific community is the use of multimedia data. In many cases, the samples used to train the models consist in images that belong to a certain kind of problem, such as medical imaging or object recognition. This field is known as \acrfull{cv}. Many world well known architectures and enormous data bases have been born during the study of this kind of problems.
	
	In the same way, audio data have been used to get conclusions from a lot of real world problems. In order to tackle the task of violent event detection it is important to decide what perspective is going to be taken into account when defining a violent event, whether an objective point of view or a more personalized standpoint according to the victim criteria. Apart from this, it is also necessary to extract the required features, that is, information from the audio signals that will allow to train the models so to get the results. However, the main work will be characterized by classifying a whole scene depending on the events this is built by. Once an action sound is categorized, it can be identified as violent by checking if it belongs to the violence definition previously defined.
		
	The different acoustic scenes that may be considered for the problem can be composed by events of different nature or those that belong to just one class. This difference may cause that the techniques utilized to address the problem can differ. As a further approach, it is interesting to find a method that can distinguish among events that come from different sources of audio.
	
\section{Regulatory framework}
\todo{Tips?}

\section{Socio-economic environment}
\todo{Tips?}
	
	
	
	