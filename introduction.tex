% !TeX spellcheck = en_GB
% INTRODUCTION text
\section{Context}

	Violence against women has been considered the most common ways of violating human rights in the whole world. This type of actions are the origin of injury and a risk factor for many psychological and physical health problems. Also, it increases the possibility of suffering from other types of health issues as physical disability, abuse of consumption of drugs and alcohol and depression \cite{Heise1999}\cite{Watts2002}. This type of violence is usually based on deep social and cultural roots and it is undoubtedly linked to unbalanced relationships between men and women in different situations and contexts, such as economics, politics and religion \cite{Blanco2004}. Also, victims are often times  women who endured violence during their childhood and felt socially isolated. %They are also characterized for a considerable economic dependency and a low educational level \cite{Ruiz-Perez2006}.
	
	One of the most common types of violence against women is the \acrfull{ipv} which can be defined as physical and emotional damage and sexual abuse by an intimate partner with a controlling attitude. Although violence between partners in a heterosexual relationship can be exerted from both sides and also my appear in homosexual cases, the most usual situation of \acrshort{ipv} is from men to women \cite{WorldHealtOrganization2012}.

	A way to retrieve information of what is happening in a certain situation is by making use of audio resources. Within this field, plenty of techniques have been developed in order to extract usable information for solving real world problems. As will be explained in detail in \ref{section:ved}, one of these tasks consists on detecting violent situations for several applications by implementing methods that take advantage of audiovisual data.
	
	In order to include the violence against women as a new branch in the violent detection, the \textit{UC3M4Safety Team} from Universidad Carlos III de Madrid has presented the project EMPATIA \footnote{\textit{protEcciónn integral de las víctimas de violencia de género Mediante comPutaciónn AfecTIva multimodAl} funded by Department  of  Research  and Innovation of Madrid Regional Authority (reference  Y2018/TCS-5046)} in which, among other types of data, acoustic information extracted from the victims' environment is to be used to perform this task \cite{UC3M4SafetyTeam2018}.

	
\section{Objectives}
\label{section:objectives}

	At the beginning of this project, the main objective of the work was to design and implement a system that allowed to detect and identify violent events from acoustic signals. In order to define a way of how to address the problem, one of the idea was to work in a sound detector. However, how do we know exactly what kind of sounds do we want to detect? Then, we chose the path of defining a system based on a classifier of acoustic events that gives as final output a binary tag identifying the input act as violent or non-violent to improve on the definition of violence. This is basically the global objective of the whole project. The following list collects the different partial objectives considered necessary to obtain the final one.
	
	\begin{itemize}
		\item To find a data resource, for example, a public database, that offers a sufficiently high number of data observations, especially, that belong to a certain type of classes that can be considered as violent in different situations.
		\item To discover a set of features or a feature space in which it is feasible to obtain a clearer separation between violent and non-violent events than the one obtained with conventional feature extraction methods.
		\item To find a customizable definition of violence that fully adapts to the victim's perception of a violent situation. % situation which hast not to be necessary considered as violence from a more objective point of view.
		\item To implement a system that allows the previous customization. %the user to transmit the whole model her own definition of violence so it can get adapted to the victim's personal situation.
		\item To find an algorithm that allows to take advantage of the new features in order to perform a multiclass classification to distinguish between violent and non-violent events.
		\item To adapt the output obtained from the multiclass task of the model to a binary classification in order to finally identify violent actions.
	\end{itemize}
		
\section{Regulatory framework}

	When paying attention to the final application of the whole EMPATIA-TC project, the regulatory framework ranges from aspects related to private data protection and the integral law of gender-based violence, specifically, the conditions related to the rights of the victims and also of the aggressors. 
	
	Since the objective of the whole project consists of collecting information from day-to-day life of the victim in order to control and record her emotional state and check for signs of violence inside her usual environment, this is a continuous access to her privacy and all types of actions that take place in every kind of situations could be recorded, as long as the victim wears the specific devices. Also, when considering evidences in trials, a good point would be to present as a valid evidence data collected with this system. This is actually a good important application of the collected data due to, in several occasions, the judicial cases become stack because of the lack of credible evidences \cite{UC3M4SafetyTeam2018}. In the case of achieving a resolution for the case, the data could also help to distinguish between the different types of sexual crimes that can be carried out \cite{Baldwin}, and so the imposed penalties for the aggressor which is different for each case.
	
	The scope of this topic within the project is so complex that a significant part of the UC3M4Safety team is specifically working towards correctly dealing with the regulatory framework. %in this field in order to establish a regulatory framework that totally adapts to these problems. 
	However, the scope of the task related to this specific project is still far from this type of considerations. Since the database used in this work is publicly available as YouTube videos, the legal requirements we deal with are nothing but the privacy policy of this website \cite{Goetzparterns}. It may have a repercussion on the development tasks when trying to download the multimedia data, since there are some videos that are no available maybe because of this reason. 

\section{Socio-economic environment}
	
	The implementation of a system like this has its most important repercussion on the increase of women empowerment and gender equality. There have been a lot of improvements in this field mainly during the last years but still the discrimination is present in all the countries in the world. 
	
	As a part of the \acrfull{sdg} as a program by the \acrfull{un}, goal 5 is the one related to the objective of \textit{achieving gender equality and empower all women and girls} \cite{UnitedNations}. There are several abuses that are still present and should be eliminated in order to aim to live in a more prosperous, sustainable and peaceful world. For example, 1 out of 5 women have declared to have suffered physical or sexual damage by an intimate partner. Also, 49 countries in the contemporary world do not count with a legislation to protect women from domestic violence. 
	
	It is mainly in this atmosphere of domestic violence in which the developed task can find a more important application within the whole project by defining a new protocol for the protection of the victims of gender-based violence including all the involved agents in order to implement a technological solution \cite{UC3M4SafetyTeam2018}.
	

	
	
	
	
	