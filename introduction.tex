% !TeX spellcheck = en_GB
% INTRODUCTION text
\section{Context}

	Violence against women remains an invisible phenomenon, deeply within the victim’s private life in most cases. It is based on deep social and cultural roots and it is undoubtedly linked to unbalanced relationships between men and women in different situations and contexts, such as economics, politics and religion. In order to prevent these conflicts, the related legislation has achieved important improvements for the last years \cite{}. \todo{CPM: falta la cita.}According to the results of most of the studies, victims can be usually defined as women who endured violence during their childhood and felt socially isolated. They are also characterized for a considerable economic dependency and a low educational level \cite{}.\todo{CPM: tienes puesto que vas a introducir unas refs. Es importante porque no se pueden mencionar estudios de esta manera "most of the studies" sin citar algunos al menos}

	With the purpose of making a difference when identifying situations showing this kind of violence and applying the knowledge and technological advances acquired during this information era, machine learning and deep learning models can collect all the available data to protect eventual victims. \todo{CPM: esta frase no se entiende, ¿quién tiene el propósito de marcar una diferencia?}
	
	The main challenge is to get to know how the victim is feeling when she is under a threatening situation, for example, if she is scared or nervous, and combine this with other variables which may play an important role in the scene and might be helpful in making a decision about its characterization. There are several factors that can be considered to achieve this task. One of them is the audio, either the victim's voice or the environmental sounds.
	
	Plenty of useful information can be extracted from the acoustic scene. %of a certain place. 
	The detection of audio events can be combined to understand % is an equally good way to define 
	what is happening in a certain moment. Once these data are collected, they can be classified in different categories and thus describe several aspects of the acoustic scene. Either based on an objective definition of gender violence or on an explanation previously obtained from a particular/specific victim, this acoustic knowledge can be interpreted as dangerous for the user. \todo{CPM: no me gusta mucho la distinción entre objetiva y particular/específica porque la primera tampoco es 'objetiva'. Se me ocurre mejor 'stereotyped' and 'personalized', por ejemplo.}

	
\section{Objectives}

	in a beginning, the main objective of the work was to design and implement a system that allowed to detect and identify violent events from acoustic signals. In order to define a way of how to address the problem, one of the idea was to work in a sound detector. However, how do we know exactly what kind of sounds do we want to detect? Then, we chose the path of defining a system based on a classifier of acoustic events that gives as final output a binary tag identifying the input act as violent or non-violent to improve on the definition of violence. This is basically the global objective of the whole project. The following list collects the different partial objectives considered necessary to obtain the final one.
	
	\begin{itemize}
		\item To find a data resource, for example, a public database, that counts with a sufficient number of data observations, especially, that belong to a certain type of classes that can be considered as violent in different situations.
		\item To discover a set of features or a feature space in which it is feasible to obtain a more clear separation between violent and non-violent events than the one obtain with more common feature extraction methods.
		\item To define an idea of violence that fully adapt to the victim situation which hast not to be necessary considered as violence from a more objective point of view.
		\item To implement a system that allows the user to transmit the whole model her own definition of violence so it can get adapted to the victim's personal situation.
		\item To find an algorithm that allows to take advantage of the new features in order to perform a multiclass classification to differentiate between violent and non-violent events.
		\item To adapt the output obtained from the multiclass task of the model to a binary classification in order to finally identify violent actions.
	\end{itemize}
		
\section{Regulatory framework}
\todo{Tips?}
\todo{CPM: por ejemplo, asuntos relacionados con la ley de protección de datos, la ley integral de violencia de género -derechos de las víctimas y los agresores-. De hecho, yo aquí diría que el marco regulatorio de la aplicación es altamente complicado y que por eso, en el proyecto hay expertas trabajando estos temas pero que tú, en particular, para el trabajo que has desarrollado (puesto que la base de datos es de acceso público) sólo has tenido que observar las instrucciones de la bd.}

\section{Socio-economic environment}
\todo{Tips?}
\todo{CPM: aquí puedes hablar del impacto socio-económico. Lo que supone empoderar a las mujeres ...Puedes hablar del los objetivos sostenibles de las Naciones Unidas (SDG 5, gender equality) y tienes un filón.}
	
	
	
	